\documentclass[]{article} % for elsevier submission
\usepackage{amsmath}
\usepackage{graphicx,psfrag,epsf}
\usepackage{enumerate}
\usepackage{natbib}
\usepackage{textcomp}
\usepackage[hyphens]{url} % not crucial - just used below for the URL
\usepackage{hyperref}
\usepackage{listings} 
\usepackage{booktabs}
\usepackage{float}
\usepackage{threeparttable}
\usepackage{threeparttablex}
%\usepackage[onehalfspacing]{setspace}
\usepackage{ebgaramond}
\usepackage[varqu,varl,var0,scaled=0.97]{inconsolata} 
\usepackage{FiraSans}
\usepackage[usenames,dvipsnames]{color}    
\usepackage[font=small, font+=singlespacing,labelfont=bf]{caption}
%\usepackage[pagewise]{lineno}
\PassOptionsToPackage{usenames,dvipsnames,svgnames}{xcolor}  
\usepackage{tikz}
\usetikzlibrary{arrows,positioning,automata}

\title{Linear Algebra Model}
\author{Stijn Masschelein}
\date{\today}

\begin{document}
\maketitle

\section{Introduction}
The goal of the paper is to provide a consistent and rigorous framework to theorise about complementarities, related theories, and the methods that are based on linear models to test for them. 

\section{The base model}

$\vec{x}$ is the $N$ choices and $\vec{z}$ is the $M$ environmental factors determining the costs and benefits of the choices. $y$ is the performance of the firm. $\vec{\beta}$ are the direct performance effects of the choices when $\vec{z} = \vec{0}$. $\vec{\delta}$ is a $M \times N$ matrix and determines the cost and benefits based on the environmental factors. $B$ is an $N$-square matrix with $0$ diagonal and captures the complementarities between the choices. Let $A = \delta^T I_N - B$. 

The production function is then the following equation.
\begin{equation}
\label{eq:production}
    y = (\vec{\beta}^T + \vec{z}^T G + \vec{\epsilon}^T) \vec{x} 
         - \frac{1}{2} \vec{x}^T A \vec{x} + \nu
\end{equation}

The optimal choices are given by 
\begin{equation}
\label{eq:optimal}
    \vec{x^*} = A^{-1}(\vec{\beta}^T + \vec{z}^T G + \vec{\epsilon}^T) 
\end{equation}

as long as $A$ is negative definite.

https://www.tulane.edu/~PsycStat/dunlap/Psyc613/RI2.html. This source makes the link between the determinant of scaled A and how much the 

\section{Extensions to the base model}

\section{Earlier Version}

\subsection{References}

\begin{enumerate}
    \item Patient zero: https://doi-org.ezproxy.library.uwa.edu.au/10.1002/smj.435
    \item The meta analysis: https://doi-org.ezproxy.library.uwa.edu.au/10.1002/smj.3042
\end{enumerate}

\subsection{Structure of a paper: Complementaries, common factors, and indices: Righting some wrongs}

\subsubsection{Theory}

$\vec{x}$ is the $N$ choices and $\vec{z}$ is the $M$ environmental factors determining the costs and benefits of the choices. $y$ is the performance of the firm. $\vec{\beta}$ are the direct performance effects of the choices when $\vec{z} = \vec{0}$. $\vec{\delta}$ are the 
$G$ is a $M \times N$ matrix and determines the cost and benefits based on the environmental factors. $B$ is an $N$-square matrix with $0$ diagonal and captures the complementarities between the choices. Let $A = \delta^T I_N - B$. 

The production function is then the following equation.
\begin{equation}
\label{eq:production}
    y = (\vec{\beta}^T + \vec{z}^T G + \vec{\epsilon}^T) \vec{x} 
         - \frac{1}{2} \vec{x}^T A \vec{x} + \nu
\end{equation}

The optimal choices are given by 
\begin{equation}
\label{eq:optimal}
    \vec{x^*} = A^{-1}(\vec{\beta}^T + \vec{z}^T G + \vec{\epsilon}^T) 
\end{equation}

as long as $A$ is negative definite.

https://www.tulane.edu/~PsycStat/dunlap/Psyc613/RI2.html. This source makes the link between the determinant of scaled A and how much the 

\subsubsection{Covariance Structure}
From \ref{eq:optimal}, it's relatively easy to derive the covariance matrix
$X^*{X^{*}}^T$ for a sample of optimal decisions. It' easy to see that the covariance depends on $Z$ as well as $A$, thus the need to control for the environment when testing 
for complementarities.

After centering of $Z$ and assuming $Z E = 0$

\begin{align}
\Sigma_x = X^TX 
     &= ((Z G + E)A^{-1})^T (Z G + E)A^{-1} \nonumber \\
     &= A^{-1} (G^T Z^T + E^T) (G Z + E) A^{-1} \nonumber \\
     &= A^{-1} (G^T \Sigma_z G + \Sigma_{\epsilon}) A^{-1} \label{eq:covariance}
\end{align}

This also means that both complementarity and the environment can be underlying reasons for the covariance. 

\subsubsection{One factor model}

See also VanderWeele for one factor model, $\eta$.

\begin{align*}
    \vec{x} &= \vec{\lambda} \eta + \vec{\delta} \\
    \vec{x} - \vec{\delta} &= \vec{\lambda} \eta \\
    \vec{\lambda}^{-1} (\vec{x} - \vec{\delta}) &= \eta \\ 
    \sum \frac{x_i - \delta_i}{\lambda_i} &= \eta \\
\end{align*}

See Bollen (1989) in Chapter 4 on identification to find closed form solutions for a 
one factor model.

The model has $2n + 1$ parameters. The covariance has $\frac{n (n+1)}{2}$ pieces of information. Assuming independence of errors.

\begin{align*}
    \vec{x} &= \vec{\lambda} \eta + \vec{\delta} \\
    E(\vec{x} \vec{x}^{T}) &= (\vec{\lambda} \eta + \vec{\delta})
        (\vec{\lambda} \eta + \vec{\delta})^{T} \\
    &= \sigma_{\eta} \vec{\lambda} \vec{\lambda}^{T} + \vec{\sigma_{\delta}} I_n \\
    E(x_i x_j) &= \sigma_{\eta} \lambda_i \lambda_j \\ 
    E(x_i x_i) &= \sigma_{\eta} \lambda_i^2 + {\sigma_{\delta}}_{ii}
\end{align*}

\subsubsection{Goldilock}

The Goldilock zone of just close enough to identify the weights but not enough to completely wash out the performance effects.

\subsubsection{An index}

Literature ignoring the issue that firm's make choices. Voluntary disclosure.

An index with weights, $\vec{w}$. This is a more general form of the common factor model.

\begin{equation}
\label{eq:index}
    \Sigma w_i x^*_i = 
    \Sigma w_i [[A^{-1}]_i (\vec{\beta}^T + \vec{z}^T G + \vec{\epsilon}^T)]
\end{equation}

\end{document}

\end{document}