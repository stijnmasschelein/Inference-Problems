\documentclass[]{article} % for elsevier submission
\usepackage{amsmath}
\usepackage{graphicx,psfrag,epsf}
\usepackage{enumerate}
\usepackage{natbib}
\usepackage{textcomp}
\usepackage[hyphens]{url} % not crucial - just used below for the URL
\usepackage{hyperref}
\usepackage{listings} 
\usepackage{booktabs}
\usepackage{float}
\usepackage{threeparttable}
\usepackage{threeparttablex}
%\usepackage[onehalfspacing]{setspace}
\usepackage{ebgaramond}
\usepackage[varqu,varl,var0,scaled=0.97]{inconsolata} 
\usepackage{FiraSans}
\usepackage[usenames,dvipsnames]{color}    
\usepackage[font=small, font+=singlespacing,labelfont=bf]{caption}
%\usepackage[pagewise]{lineno}
\PassOptionsToPackage{usenames,dvipsnames,svgnames}{xcolor}  
\usepackage{tikz}
\usetikzlibrary{arrows,positioning,automata}

\title{Mediation}
\author{Stijn Masschelein}
\date{December 2020}


\begin{document}
\maketitle

\section{Introduction}
The goal of the paper is to provide a consistent and rigorous framework to theorise about complementarities, related theories, and the methods that are based on linear models to test for them. 

changes

\section{The base model}

The $k$ choices for company $i$ are the vector $x_i$. The $p$ environmental factors are in $z_i$. The linear effects are in $b_{k \times 1}$. The quadratic (complementarity) effects are in $A_{k \times k}$. The contingency effects are in $G_{p \times k}$. The performance is $y_i$. 

\begin{aligned}
\label{eq:management-function}
y_i &= (b + z_i G) x_i - \frac{1}{2} x_i' A x_i \\
&= (b_{Zi} + \epsilon_i) x_i - \frac{1}{2} x_i' A x_i
\end{aligned}

When $A$ is positive definite, the optimal choice for $x_i$ is given by

\begin{aligned}
\label{eq:optimal}
Ax_i &= b_{Zi} + \epsilon_i \\
x_i^* &= A^{-1} (b_{Zi} + \epsilon_i)
\end{aligned}

The positive definite condition implies that the dyadic complementarities are not too big as to overwhelm the increasing marginal costs. If this condition is violated, the optimal choice is a corner solution. 

The equation \ref{eq:optimal} shows that the optimal choice is a funtion of the joint contingency factors and the individual contingency factors. These are the only causal factors in the theoretical model. There is no latent system variable. 


\section{Extensions to the base model}

\end{document}
