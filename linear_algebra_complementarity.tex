\documentclass[]{article} % for elsevier submission
\usepackage{amsmath}
\usepackage{graphicx,psfrag,epsf}
\usepackage{enumerate}
\usepackage{natbib}
\usepackage{textcomp}
\usepackage[hyphens]{url} % not crucial - just used below for the URL
\usepackage{hyperref}
\usepackage{listings} 
\usepackage{booktabs}
\usepackage{float}
\usepackage{threeparttable}
\usepackage{threeparttablex}
%\usepackage[onehalfspacing]{setspace}
\usepackage{ebgaramond}
\usepackage[varqu,varl,var0,scaled=0.97]{inconsolata} 
\usepackage{FiraSans}
\usepackage[usenames,dvipsnames]{color}    
\usepackage[font=small, font+=singlespacing,labelfont=bf]{caption}
%\usepackage[pagewise]{lineno}
\PassOptionsToPackage{usenames,dvipsnames,svgnames}{xcolor}  
\usepackage{tikz}
\usetikzlibrary{arrows,positioning,automata}

\title{Mediation}
\author{Stijn Masschelein}
\date{December 2020}


\begin{document}
\maketitle

\section{Introduction}
The goal of the paper is to provide a consistent and rigorous framework to theorise about complementarities, related theories, and the methods that are based on linear models to test for them. 

\section{The base model}

\begin{equation}
\label{eq:management-function}
y_i = x_i 
\end{equation}

\section{Extensions to the base model}

\end{document}